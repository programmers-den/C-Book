\documentclass[../main.tex]{subfiles}

% Flowchart Symbols
%\usetikzlibrary{shapes.geometric, arrows}

\begin{document}
\chapter{Thinking Like a C Programmer}

\vspace{0.2cm}
\begin{center}
	\textit{
		The good news about computers is that they do what you tell them to do.\\
		The bad news is that they do what you tell them to do.
	}\\\vspace{0.5cm}
	Ted Nelson
\end{center}
\vspace{0.75cm}

\section{Understanding the Computer}
A computer, in essence, is a system that can take some input data, do some math,
remember the result, and recall the results later. The precise origin of the data
does not particularly matter - The data might come from memory, user input,
the Internet, a sensor, or some other source.

There are many different models to aid in understanding how a computer program works,
but the most effective is a simple flowchart model that will be expanded upon later.

\begin{obifig}{caption=A computer doing work}
	\begin{tikzpicture}
		\node[io]      (input)  at (0, 0) {Input};
		\node[process] (prog)   at (5, 0) {Computer};
		\node[io]      (output) at (10, 0) {Output};
		
		\draw[thick,->] (input) -- (prog);
		\draw[thick,->] (prog)  -- (output);
		\draw[thick]    (prog.north) arc (0:90:0.5) -- (3.3, 1) arc (90:270:0.5) (prog.west);
	\end{tikzpicture}
\end{obifig}

A computer can be modeled as a black box that does processing on a data stream,
as shown above, but what is it that the computer actually \textit{does}? And why does
the flowchart loop back on itself in the \textit{Computer} stage? The answer to both of these questions can be found by investigating State Machines.

\subsection{The State Machine}
The State Machine is a model that allows introspection into what the computer is
actually doing. In it, a computer program is segmented into multiple, discrete
states. These states have conditions, that if met, will trigger the system to perform
an action and move to a different state.

Examining a simple example will help illustrate how these transitions work.
Consider a basic vending machine, only selling one product. It has two basic states:
(1) accepting payment and
(2) product selection.

The start state is also marked, as otherwise the machine starting state is left
undefined. Undefined behavior is a Bad Thing\texttrademark, and leads to
potentially unexpected outcomes, such as the vending machine starting in the
\textit{Select Product} state. Such a machine could be exploited by would-be
thieves by simply unplugging the machine and plugging it back in!

\begin{obifig}{caption=A Basic State Machine}
	\begin{tikzpicture}
		\node[state,fill=red!30] (start) at (0, 2) {Start};
		\node[state] (pay)     at (0, 0) {Accepting \\ Payment};
		\node[state] (disp)    at (7, 0) {Select \\ Product};
		
		\draw[thick, ->] (start) -- (pay);
		
		\draw[thick, ->] (pay) edge[above, bend left=15]
			node {Insert Currency} (disp);
		\draw[thick, ->] (disp) edge[below, bend left=15]
			node {Product Selected} (pay);
		
		\draw[thick, ->] (pay) edge[loop left,align=center,looseness=4]
			node {Product\\Selected} (pay);
		\draw[thick, ->] (disp) edge[loop right,align=center,looseness=4]
			node {Insert\\Currency} (disp);

	\end{tikzpicture}
\end{obifig}

In the \textit{Accepting Payment} state, the machine sits and waits for the customer
to insert payment. Any time one of the sensors determines that payment has been made,
the machine transitions to the \textit{Select Product} state. While in the
\textit{Select Product} state, the machine waits for the customer to select a product,
dispenses a product and transitions back to the \textit{Accepting Payment} state.

If the machine is waiting for one type of input, it ignores/rejects another type of
input. For example, in the \textit{Accepting Payment} state, the machine ignores
any action the customer takes to select a product. Similarly, when in the
\textit{Select Product} state, the machine rejects any inserted currency.

A state machine like the one presented above is also known as a Finite State Machine,
as the number states is finite. These state machines can also be described in a table
called a state-transition table.

\begin{obitab}{size=\small, cols=|l|l|l|l|,
	caption=Vending Machine State Transition Table}

	\hline
	\rowcolor{black}
		\othead{Current State}
		& \othead{Input}
		& \othead{Next State}
		& \othead{Action} \\
	\hline
	
	\multirow{2}{*}{Accepting Payment}
		& Insert Currency & Select Product & Allows user to select a product\\
		\cline{2-4}
		& Product Selected & Accepting Payment & Does Nothing \\
	\hline
	
	\multirow{2}{*}{Select Product}
		& Insert Currency & Select Product & Does Nothing \\
		\cline{2-4}
		& Product Selected & Accepting Payment & Dispenses the selected product\\
	\hline
\end{obitab}

The design of such state machines can grow quite
complex, however, as any information the machine needs to store must be a unique
state. To demonstrate, imagine if the machine needed to keep track of different
denominations of currency or make change.

\section{Problem Solving}
\subsection{Programming or Software Engineering}
There are entire books written on the subject of project management and problem 
solving with much more depth than can be described here. For a deep dive into the
theory and methodology of solving problems, there are many resources available: video 
lecture series, articles, blog posts, books, and others. As such, the overall goal
here is not to dive into an expert-level description of the problem solving process.
The purpose here is to explore the mindset and thought process of a programmer
approaching a problem.

Many will draw a distinction between a programmer and a software engineer; the sole 
role of a programmer is to write code, while a software engineer manages the process 
of architecture and design. This is a false dichotomy, however. Programmers, i.e. 
people who write code, are necessarily involved in the problem solving process and
need to make design decisions as they solve the problem at hand. This is especially the case for hobbyists and those on single-person teams.

Every stage of writing code, from the first line to the last, by its nature involves
making design decisions. Should the code be unit tested? How will this module be
structured? What language should be used? How will the problem be broken down?

While it is true that "correctness" is something that can be specified, and that 
there are some problems that are so simple as to elicit the same approach by two 
different programmers, any problem complex enough as to be interesting will require 
answering these questions and many more. Just as in many other fields, there is not 
usually one correct approach to solving the problem at hand. Some design approaches 
might be faster to run; some might be easier to implement, or perhaps easier to 
maintain. But consciously or not, an approach was chosen and a design decision made.

\subsection{The Problem Solving Process}
In a general sense, it is the role of the programmer to solve a problem with code.
The problem could be a contrived one, such as "this process runs 5\% too slow;" an
end-user goal, such as "listen to music without an Internet connection;"
an analysis of some kind, such as "run a simulation model and compare results with
measured data;" or some other useful action.

The first step is to identify the problem that is needing solving. Part of this 
identification involves a thorough definition of the problem (i.e. the Problem 
Statement). An unfortunate reality of most problem statements is that they are often
overly broad. Users, management, and even other programmers often do not understand 
the complexity involved in developing a program, and so, underestimate the cost of
adding additional features. This is especially true if they lack specific domain
knowledge.

For example, consider the problem statement "the answer to life, the
universe and everything". Where would modeling this problem even begin? As stated, 
the problem is quite broad, and would pose significant difficulty in implementation.
Working to narrow the problem down to necessary items only might reveal that the
actual problem the user needs solving is to perform basic math, i.e. a four-function 
calculator: "multiply, divide, add or subtract two numbers; then display the result."

Once the problem is identified and clearly stated, the problem is simplified further 
into multiple, smaller sub-problems. Each of these can be solved independently, then 
put together to solve the original problem. These sub-problems can often be further
broken down into smaller and smaller problems.

\begin{obifig}{caption=The Problem Solving Process}
	\begin{tikzpicture}
		\node[io]      (pstatement)   at (0, 10) {Problem Statement};
		
		\node[process] (sproblem1)    at (-5, 8) {Sub Problem 1};
		\node[process] (sproblem2)    at (2, 8)  {Sub Problem 2};
		
		
		\node[process] (sproblem3) at (0, 6)     {Sub Problem 3};
		\node[process] (sproblem4) at (5, 6)     {Sub Problem 4};
		
		\node[process] (sproblemsol1) at (-5, 4) {Solved Sub Problem 1};
		\node[process] (sproblemsol3) at (0, 4)  {Solved Sub Problem 3};
		\node[process] (sproblemsol4) at (5, 4)  {Solved Sub Problem 4};
		
		\node[io]      (solved)       at (0, 2)  {Solution};
		
		\node[startstop] (solution)   at (0, 0)  {Solved Problem};
		
		\draw[thick,->] (pstatement) -- (sproblem1);
		\draw[thick,->] (pstatement) -- (sproblem2);
		
		\draw[thick,->] (sproblem2) -- (sproblem3);
		\draw[thick,->] (sproblem2) -- (sproblem4);
		
		\draw[thick,->] (sproblem1) -- (sproblemsol1);
		\draw[thick,->] (sproblem3) -- (sproblemsol3);
		\draw[thick,->] (sproblem4) -- (sproblemsol4);
		
		
		\draw[thick,->] (sproblemsol1) -- (solved);
		\draw[thick,->] (sproblemsol3) -- (solved);
		\draw[thick,->] (sproblemsol4) -- (solved);
		
		\draw[thick,->] (solved) -- (solution);
	\end{tikzpicture}
\end{obifig}

\subsection{A Sample Problem}
To demonstrate how the process works, consider the calculator example mentioned in
previous section. The user wants to be able to take two numbers, and either multiply,
divide, add, or subtract them. That is the problem statement, already complete!

Next step - simplify the problem into smaller, sub-problems. The way the problem is
stated already lends itself nicely to how this might be split:
\begin{itemize}
	\item Multiplying two numbers
	\item Dividing two numbers
	\item Adding two numbers
	\item Subtracting two numbers
\end{itemize}

Assume, for a moment that each of these sub-problems are solved. How might the 
process of taking each of these solutions and composing it into the overall problem
solution be completed? The answer to this question should serve as a check to ensure
all parts of the problem have been modeled. If a solution cannot be formed from
the sub-solutions, there is likely a hidden sub-problem that remains to be explored.

To model this solution, a flowchart will be created. This shows both the order of
operations within the system, as well as how data flows from one solution to another. Given the solutions to the sub-problems mentioned as building blocks, a calculator program may now be modeled.

\begin{obifig}{caption=Sample Problem Modeling}
	\begin{tikzpicture}
		\node[process] (process)   at (0, 3) {Calculate Data};

		\node[process] (subp1)   at (-6, 0) {Multiply $A \times B$};
		\node[process] (subp2)   at (-2.1, 0) {Divide $A \div B$};
		\node[process] (subp3)   at (1.8, 0) {Add $A + B$};
		\node[process] (subp4)   at (5.7, 0) {Subtract $A - B$};
		
		\draw[thin, opacity=0.75] ([xshift=-.4cm,yshift=.4cm]subp1.west) --
			([xshift=-.4cm,yshift=1cm]subp1.west) --
			([xshift=.4cm,yshift=1cm]subp4.east) --
			([xshift=.4cm,yshift=.4cm]subp4.east);
			
		\draw[thin, opacity=0.75] ([xshift=-.4cm,yshift=-.4cm]process.west) --
			([xshift=-.4cm,yshift=-1cm]process.west) --
			([xshift=.4cm,yshift=-1cm]process.east) --
			([xshift=.4cm,yshift=-.4cm]process.east);
			
		\draw[thick,->] ([yshift=-.7cm]process.south) -- ([yshift=-1.3cm]process.south);
		
		\draw[thin, opacity=0.75] ([xshift=-.4cm,yshift=-.4cm]subp1.west) --
			([xshift=-.4cm,yshift=-1cm]subp1.west) --
			([xshift=.4cm,yshift=-1cm]subp4.east) --
			([xshift=.4cm,yshift=-.4cm]subp4.east);		
			
		\node[startstop] (start) at (-6, -3.5) {Start};
		\node[io]        (getA)  at (-6, -5.5) {?};
		
		\node[decision]  (whichOp) at (0, -3.5) {Op?};
		
		\node[process] (opMul)   at (5.8, -3.5)  {Multiply $A \times B$};
		\node[process] (opDiv)   at (3.9, -4.75)  {Divide $A \div B$};
		\node[process] (opAdd)   at (2.4, -6)   {Add $A + B$};
		\node[process] (opSub)   at (2, -7.5)   {Subtract $A - B$};
		
		\node[startstop] (end)   at (5.8, -7.5) {Solution};

		\draw[thick,->] (start) -- (getA);
		
		\draw[thick,->,dotted] (getA) -- (whichOp);
		
		\draw[thick,->] (whichOp) -- (opMul);
		\draw[thick,->] (whichOp) -- (opDiv);
		\draw[thick,->] (whichOp) -- (opAdd);
		\draw[thick,->] (whichOp) -- (opSub.155);
		
		\draw[thick,->] (opMul) -- (end);
		\draw[thick,->] (opDiv) -- (end);
		\draw[thick,->] (opAdd) -- (end);
		\draw[thick,->] (opSub) -- (end);
		
		\draw[thin, opacity=0.75] ([xshift=-.4cm,yshift=.4cm]start.west) --
			([xshift=-.4cm,yshift=1cm]start.west) --
			([xshift=.4cm,yshift=1cm]opMul.east) --
			([xshift=.4cm,yshift=.4cm]opMul.east);
			
		\draw[thick,->] (0, -1.3) -- (0, -2.3); 

	\end{tikzpicture}
\end{obifig}


Oh, no! An impasse has been encountered! Several important pieces have been left off.
How will the program know which operation the user wanted to perform? How will it
determine the two numbers ($A$ and $B$) that it should use as its operands? To fully 
solve this problem, the program will also need to solve these sub-problems as well.

\begin{itemize}
	\item Accepting a number from the user
	\item Accepting an operation from the user
\end{itemize}

With these additions to the set of sub-problems and solutions, a complete flowchart
can be created:

\begin{obifig}{caption=Sample Problem Modeling - Full Solution}
	\begin{tikzpicture}
%		\node[process] (process) at (0, 9)   {Calculate Data};
%		
%		\draw[thin, opacity=0.75] ([xshift=-.4cm,yshift=-.4cm]process.west) --
%			([xshift=-.4cm,yshift=-1cm]process.west) --
%			([xshift=.4cm,yshift=-1cm]process.east) --
%			([xshift=.4cm,yshift=-.4cm]process.east);
%			
%	
%		
%		\node[process] (subp1)   at (-6, 6)  {Multiply $A \times B$};
%		\node[process] (subp2)   at (-2, 6)  {Divide $A \div B$};
%		\node[process] (subp3)   at (2, 6)   {Add $A + B$};
%		\node[process] (subp4)   at (6, 6)   {Subtract $A - B$};
%		
%		\node[process] (subp5)   at (-3, 4.5) {Get a Number};
%		\node[process] (subp6)   at (3, 4.5)  {Get an Operation};
%		
%		\draw[thin, opacity=0.75] ([xshift=-.4cm,yshift=.4cm]subp1.west) --
%			([xshift=-.4cm,yshift=1cm]subp1.west) --
%			([xshift=.4cm,yshift=1cm]subp4.east) --
%			([xshift=.4cm,yshift=.4cm]subp4.east);
%			
%		\draw[thick,->] ([yshift=-.7cm]process.south) --  ([yshift=-1.3cm]process.south);	
%		
%		\draw[thin, opacity=0.75] ([xshift=-.4cm,yshift=-1.9cm]subp1.west) --
%			([xshift=-.4cm,yshift=-2.5cm]subp1.west) --
%			([xshift=.4cm,yshift=-2.5cm]subp4.east) --
%			([xshift=.4cm,yshift=-1.9cm]subp4.east);
%
%		
		\node[startstop] (start) at (-6, 1) {Start};
		\node[io]        (getA)  at (-6, -1) {Get a Number (A)};
		\node[io]        (getB)  at (-6, -3) {Get a Number (B)};
		\node[io]        (getOp) at (-2, -1) {Get an Operation};
		
		\node[decision]  (whichOp) at (0, 1) {Op?};
		
		\node[process] (opMul)   at (6, 1)  {Multiply $A \times B$};
		\node[process] (opDiv)   at (4, -.25)  {Divide $A \div B$};
		\node[process] (opAdd)   at (2.5, -1.5)   {Add $A + B$};
		\node[process] (opSub)   at (2, -3)   {Subtract $A - B$};
		
		\node[startstop] (end)   at (6, -3) {Solution};

		\draw[thick,->] (start) -- (getA);
		\draw[thick,->] (getA) -- (getB);
		\draw[thick,->] (getB) -- (getOp.200);
		
		\draw[thick,->] (getOp.120) -- (whichOp);
		
		\draw[thick,->] (whichOp) -- (opMul);
		\draw[thick,->] (whichOp) -- (opDiv);
		\draw[thick,->] (whichOp) -- (opAdd);
		\draw[thick,->] (whichOp) -- (opSub.155);
		
		\draw[thick,->] (opMul) -- (end);
		\draw[thick,->] (opDiv) -- (end);
		\draw[thick,->] (opAdd) -- (end);
		\draw[thick,->] (opSub) -- (end);
		
%		\draw[thin, opacity=0.75] ([xshift=-.4cm,yshift=.4cm]start.west) --
%			([xshift=-.4cm,yshift=1cm]start.west) --
%			([xshift=.4cm,yshift=1cm]opMul.east) --
%			([xshift=.4cm,yshift=.4cm]opMul.east);
%			
%		\draw[thick,->] (0, 3.2) -- (0, 2.2); 
	\end{tikzpicture}
\end{obifig}

\subsection{}

\section{Programming Without Code}


\end{document}